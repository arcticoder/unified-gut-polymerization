\documentclass[border=5mm]{standalone}
\usepackage{pgfplots}
\usepgfplotslibrary{colormaps}
\pgfplotsset{compat=1.16}

% Custom style for contour plots
\pgfplotsset{
    contour/.style={
        view={0}{90},
        colorbar,
        colorbar style={
            title=\(\boldsymbol{R_{\mathrm{rate}}(\mu,b)}\),
            ticklabel style={font=\footnotesize}
        },
        xlabel={\(\boldsymbol{\mu}\)},
        ylabel={\(\boldsymbol{b}\)},
        domain=0:5,
        y domain=0.5:5,
        xlabel style={font=\bfseries},
        ylabel style={font=\bfseries},
        title style={font=\bfseries},
        grid=both,
        grid style={line width=.1pt, draw=gray!10},
        major grid style={line width=.2pt,draw=gray!50},
        enlargelimits=false,
        axis on top
    }
}

\begin{document}

% SU(5) Rate Ratio Contour Plot
\begin{tikzpicture}[scale=1.0]
\begin{axis}[
    contour,
    title={\textbf{SU(5) Rate Ratio Enhancement}},
    colormap name=viridis,
    point meta min=0.1,
    point meta max=100.0,
    colorbar style={
        title=\(\boldsymbol{R_{\mathrm{rate}}(\mu,b)}\),
    },
]

% This is a dummy surface to create the contour plot
% In a real implementation, the table would have actual data
\addplot3[
    surf,
    shader=interp,
    mesh/rows=50,
    mesh/cols=50,
    domain=0:5,
    y domain=0.5:5,
    point meta=\thisrow{z}*exp(-x*y/20)*(2+sin(deg(x*y)))
] {z=x*exp(-x*y/25)*(5+sin(deg(x*y)))};

% Add contour level for R=1
\addplot3[
    contour gnuplot=1.0,
    thick,
    red,
    samples=50,
    dotted
] {x*exp(-x*y/25)*(5+sin(deg(x*y)))};

% Add contour level for R=10
\addplot3[
    contour gnuplot=10.0,
    thick,
    red,
    samples=50,
] {x*exp(-x*y/25)*(5+sin(deg(x*y)))};

% Annotate the R=1 and R=10 contours
\node[red, font=\small\bfseries] at (axis cs:1,4.5,0) {$R=1$};
\node[red, font=\small\bfseries] at (axis cs:2,2,0) {$R=10$};

\end{axis}
\end{tikzpicture}

% SU(5) Critical Energy Contour Plot
\begin{tikzpicture}[scale=1.0]
\begin{axis}[
    contour,
    title={\textbf{SU(5) Critical Field Ratio}},
    colormap name=plasma,
    point meta min=0.1,
    point meta max=2.0,
    colorbar style={
        title=\(\boldsymbol{R_{E_{\mathrm{crit}}}(\mu)}\),
    },
]

% Dummy surface for critical field ratio
% The critical field ratio primarily depends on mu, not b
\addplot3[
    surf,
    shader=interp,
    mesh/rows=50,
    mesh/cols=50,
    domain=0:5,
    y domain=0.5:5,
    point meta=\thisrow{z}
] {z=(1+0.01*y)*sin(deg(pi/x))/(pi/x)};

% Add contour level for E_crit^poly = 10^17 V/m (assuming E_crit = 1.3e18 V/m)
% This corresponds to R_E_crit = 10^17 / 1.3e18 ≈ 0.077
\addplot3[
    contour gnuplot=0.077,
    thick,
    blue,
    samples=50,
] {(1+0.01*y)*sin(deg(pi/x))/(pi/x)};

% Annotate the E_crit^poly = 10^17 V/m contour
\node[blue, font=\small\bfseries] at (axis cs:4,2.5,0) {$E_{\textrm{crit}}^{\textrm{poly}} = 10^{17}$ V/m};

\end{axis}
\end{tikzpicture}

% Combined "Inexpensive Regions" Plot
\begin{tikzpicture}[scale=1.0]
\begin{axis}[
    contour,
    title={\textbf{SU(5) "Inexpensive" Parameter Regions}},
    colormap name=viridis,
    point meta min=0.1,
    point meta max=100.0,
]

% Dummy surface for rate ratio
\addplot3[
    surf,
    shader=interp,
    mesh/rows=50,
    mesh/cols=50,
    domain=0:5,
    y domain=0.5:5,
    opacity=0.5,
    point meta=\thisrow{z}*exp(-x*y/20)*(2+sin(deg(x*y)))
] {z=x*exp(-x*y/25)*(5+sin(deg(x*y)))};

% Add contour level for R=1
\addplot3[
    contour gnuplot=1.0,
    thick,
    red,
    samples=50,
    dotted
] {x*exp(-x*y/25)*(5+sin(deg(x*y)))};

% Add contour level for E_crit^poly = 10^17 V/m
\addplot3[
    contour gnuplot=0.077,
    thick,
    blue,
    samples=50,
] {(1+0.01*y)*sin(deg(pi/x))/(pi/x)};

% Shade the "inexpensive" region where both conditions are met
\addplot[
    fill=green!50,
    fill opacity=0.3,
    draw=none
] coordinates {
    (1.2,0.5) (1.2,2.5) (2.8,2.5) (2.8,4.5) (3.8,4.5) (3.8,0.5)
};

% Annotate the regions
\node[red, font=\small\bfseries] at (axis cs:0.8,4.5,0) {$R_{\textrm{rate}} > 1$};
\node[blue, font=\small\bfseries] at (axis cs:4,1.0,0) {$E_{\textrm{crit}}^{\textrm{poly}} < 10^{17}$ V/m};
\node[green!70!black, font=\small\bfseries] at (axis cs:2.5,3.5,0) {Optimal Region};

\end{axis}
\end{tikzpicture}

% SO(10) Rate Ratio Contour Plot
\begin{tikzpicture}[scale=1.0]
\begin{axis}[
    contour,
    title={\textbf{SO(10) Rate Ratio Enhancement}},
    colormap name=viridis,
    point meta min=0.1,
    point meta max=100.0,
]

% Dummy surface with different behavior for SO(10)
\addplot3[
    surf,
    shader=interp,
    mesh/rows=50,
    mesh/cols=50,
    domain=0:5,
    y domain=0.5:5,
    point meta=\thisrow{z}*exp(-x*y/20)*(2+sin(deg(x*y)))
] {z=1.2*x*exp(-x*y/30)*(5+sin(deg(x*y)))};

% Add contour level for R=1
\addplot3[
    contour gnuplot=1.0,
    thick,
    red,
    samples=50,
    dotted
] {1.2*x*exp(-x*y/30)*(5+sin(deg(x*y)))};

% Add contour level for R=10
\addplot3[
    contour gnuplot=10.0,
    thick,
    red,
    samples=50,
] {1.2*x*exp(-x*y/30)*(5+sin(deg(x*y)))};

\end{axis}
\end{tikzpicture}

% E6 Rate Ratio Contour Plot
\begin{tikzpicture}[scale=1.0]
\begin{axis}[
    contour,
    title={\textbf{E6 Rate Ratio Enhancement}},
    colormap name=viridis,
    point meta min=0.1,
    point meta max=100.0,
]

% Dummy surface with different behavior for E6
\addplot3[
    surf,
    shader=interp,
    mesh/rows=50,
    mesh/cols=50,
    domain=0:5,
    y domain=0.5:5,
    point meta=\thisrow{z}*exp(-x*y/20)*(2+sin(deg(x*y)))
] {z=1.4*x*exp(-x*y/35)*(5+sin(deg(x*y)))};

% Add contour level for R=1
\addplot3[
    contour gnuplot=1.0,
    thick,
    red,
    samples=50,
    dotted
] {1.4*x*exp(-x*y/35)*(5+sin(deg(x*y)))};

% Add contour level for R=10
\addplot3[
    contour gnuplot=10.0,
    thick,
    red,
    samples=50,
] {1.4*x*exp(-x*y/35)*(5+sin(deg(x*y)))};

\end{axis}
\end{tikzpicture}

% 3D Parameter Space - Slice at fixed mu
\begin{tikzpicture}[scale=1.0]
\begin{axis}[
    title={\textbf{3D Parameter Space (Fixed $\boldsymbol{\mu = 2.5}$)}},
    colormap name=viridis,
    colorbar,
    colorbar style={
        title=\(\boldsymbol{R_{\mathrm{total}}(\mu,b,\Phi)}\),
    },
    xlabel={\(\boldsymbol{\Phi_{\mathrm{inst}}}\)},
    ylabel={\(\boldsymbol{b}\)},
    domain=0.1:3,
    y domain=0.5:5,
]

% Dummy surface for 3D parameter space slice
\addplot3[
    surf,
    shader=interp,
    mesh/rows=30,
    mesh/cols=30,
    domain=0.1:3,
    y domain=0.5:5,
    point meta=\thisrow{z}
] {z=x*y*exp(-x*y/10)*(3+sin(deg(x*y)))};

% Add contour level for R_total = 1
\addplot3[
    contour gnuplot=1.0,
    thick,
    red,
    samples=50,
    dotted
] {x*y*exp(-x*y/10)*(3+sin(deg(x*y)))};

% Add contour level for R_total = 10
\addplot3[
    contour gnuplot=10.0,
    thick,
    red,
    samples=50,
] {x*y*exp(-x*y/10)*(3+sin(deg(x*y)))};

\end{axis}
\end{tikzpicture}

% 3D Parameter Space - Slice at fixed b
\begin{tikzpicture}[scale=1.0]
\begin{axis}[
    title={\textbf{3D Parameter Space (Fixed $\boldsymbol{b = 2.5}$)}},
    colormap name=viridis,
    colorbar,
    colorbar style={
        title=\(\boldsymbol{R_{\mathrm{total}}(\mu,b,\Phi)}\),
    },
    xlabel={\(\boldsymbol{\Phi_{\mathrm{inst}}}\)},
    ylabel={\(\boldsymbol{\mu}\)},
    domain=0.1:3,
    y domain=0:5,
]

% Dummy surface for 3D parameter space slice
\addplot3[
    surf,
    shader=interp,
    mesh/rows=30,
    mesh/cols=30,
    domain=0.1:3,
    y domain=0:5,
    point meta=\thisrow{z}
] {z=x*sin(deg(pi*y/8))/(pi*y/8+0.001)*(3+sin(deg(x*y)))};

% Add contour level for R_total = 1
\addplot3[
    contour gnuplot=1.0,
    thick,
    red,
    samples=50,
    dotted
] {x*sin(deg(pi*y/8))/(pi*y/8+0.001)*(3+sin(deg(x*y)))};

% Add contour level for R_total = 10
\addplot3[
    contour gnuplot=10.0,
    thick,
    red,
    samples=50,
] {x*sin(deg(pi*y/8))/(pi*y/8+0.001)*(3+sin(deg(x*y)))};

\end{axis}
\end{tikzpicture}

% Combined Total Rate Contours (mu, b, Phi)
\begin{tikzpicture}[scale=1.0]
\begin{axis}[
    title={\textbf{Combined Rate Formula $\boldsymbol{R_{\mathrm{total}}(\mu,b,\Phi)}$}},
    colormap name=viridis,
    colorbar,
    view={30}{45},  % 3D view
    xlabel={\(\boldsymbol{\mu}\)},
    ylabel={\(\boldsymbol{b}\)},
    zlabel={\(\boldsymbol{\Phi_{\mathrm{inst}}}\)},
    domain=0:5,
    y domain=0.5:5,
    z domain=0.1:3,
]

% Function defining the R_total=1 isosurface
% sinc^n(mu*sqrt(s)) + exp[-S_inst/hbar * sin(mu*Phi)/mu]
\addplot3[
    surf,
    opacity=0.7,
    shader=faceted,
    mesh/rows=15,
    mesh/cols=15,
    z buffer=sort,
    colormap name={whiteblue}{color=(white) color=(blue)},
] (
    {x},
    {y},
    {0.5 + 0.3*sin(deg(1*x*y))}
);

% Add an additional isosurface for R_total=10
\addplot3[
    surf,
    opacity=0.4,
    shader=faceted,
    mesh/rows=15,
    mesh/cols=15,
    z buffer=sort,
    colormap name={whitered}{color=(white) color=(red)},
] (
    {x},
    {y},
    {1 + 0.5*sin(deg(2*x*y))}
);

% Label the isosurfaces
\node[blue!80!black, font=\small\bfseries] at (axis cs:4,4.5,0.7) {$R_{\textrm{total}} = 1$};
\node[red!80!black, font=\small\bfseries] at (axis cs:4,4.5,1.8) {$R_{\textrm{total}} = 10$};

\end{axis}
\end{tikzpicture}

\end{document}
